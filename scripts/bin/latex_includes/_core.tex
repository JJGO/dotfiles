%!TEX root = report.tex
%% Jose Javier Gonzalez Ortiz %%
%% Core default Configuration %%
%% 2016-07-07                 %%

%%%%%%%%%%%%
% PACKAGES %
%%%%%%%%%%%%

\usepackage[T1]{fontenc} % Use 8-bit encoding that has 256 glyphs
\usepackage[utf8]{inputenc} % For Spanish characters
\usepackage[spanish, english]{babel}  % English localisation
% \usepackage[spanish, es-noshorthands,es-tabla]{babel}               % Spanish hyphenation and document rules


% =========== General Formatting =============
\usepackage[left=2.5cm, right=2.5cm, top=2.5cm, bottom=2cm]{geometry} % Margin sizes
\usepackage{microtype} % Slightly tweak font spacing for aesthetics
\usepackage{fancyhdr} % Allows for nice header and footer
\usepackage{sectsty} % Allows customizing section commands
\usepackage{titlesec} % Allows customization of section titles
\usepackage{appendix} % Enables appendices
\usepackage{enumerate} % Custom numerate, useful for i,ii,iii... I,II,III...
\usepackage{relsize} % Defines several commands to set font sizes relative to the current size. Useful for enlarging math
\usepackage{setspace} % Sup­port for set­ting the spac­ing be­tween lines in a doc­u­ment. \sin­glespac­ing, \one­half­s­pac­ing, and \dou­blespac­ing
\usepackage{changepage} % To change relative geometry of particular pages
\usepackage[usenames,dvipsnames]{xcolor} % Required for custom colors

% =========== Linking =============
\usepackage{hyperref} % For hyperlinks in the PDF
\hypersetup{
    colorlinks = false,
    linkcolor = red,
    urlcolor  = blue,
    citecolor = green,
    filecolor = cyan,
    anchorcolor = black
}
% =============== Math ===============
\usepackage{amsmath} % Standard math packages
\usepackage{amsthm} % Math Theorems
\usepackage{amssymb} % Math Symbols
\usepackage{amsfonts} % Math Fonts
\usepackage{upgreek} % Nice sigma => \upsigma
\usepackage{array} % Enables array features
\usepackage{siunitx} % For SI Unit easy formatting

% =============== Figures ===============
\usepackage{graphicx} % Image insertion.
\usepackage{float} % % Required for tables and figures in the multi-column environment - they need to be placed in specific locations with the [H] (e.g. \begin{table}[H])
\usepackage{wrapfig} % Al­lows fig­ures or ta­bles to have text wrapped around them
\usepackage{rotating} % Allows for sideways figure with twoside recognition

% =============== Tables ===============
\usepackage{booktabs} % Horizontal rules in tables
\usepackage{multirow} % Combined rows in tables
\usepackage{multicol} % Combined columns in tables
\usepackage{colortbl} % Color cells
\usepackage{longtable} % Tables than span multipages

% ============ Bibliography ============
\usepackage{csquotes}% Recommended package for biblatex
% Use biblatex as the bibliography package.
% The specific settings emulate the tradional bibtex alpha style
\usepackage[backend=bibtex]{biblatex}


% ============ Glossaries ============
\usepackage[acronym]{glossaries} % Very good package to create glossaries, acronym lists or symbols (nomenclature) lists.

% =============== Footnotes ===============
\usepackage{footnote}
\makesavenoteenv{figure} % Footnotes in Figures are displayed correctly
\makesavenoteenv{table} % Footnotes in Tables are displayed correctly

% =============== Other ===============
\usepackage{datetime} % Date-Time formatting
% \usepackage{ulem} % For strikethrough text \st{}
% \normalem %Restores the emphasis to italics after ulem
\usepackage{textcomp} %Text Com­pan­ion fonts

\usepackage{pdfpages} % Insert pdfs
\usepackage{lipsum} % Used for inserting dummy 'Lorem ipsum' text into the template
\usepackage{lettrine} % The lettrine is the first enlarged letter at the beginning of the text
\usepackage{pdflscape} % Individual horizontal pages
%\usepackage[space]{grffile} % insert files with spaces
%\usepackage{xargs} % Expanded arguments features
%\usepackage{fix-cm} % Computer-Modern at arbitrarysizes
\usepackage{eurosym} % Eurosymbol
%\usepackage{watermark} % To use display water marks in the document

%%%%%%%%%%%%
% CAPTIONS %
%%%%%%%%%%%%
\usepackage[hang, small,labelfont=bf,up]{caption} % Custom captions under/above floats in tables or figures
\usepackage{subcaption} % For custom caption environments
    % \numberwithin{equation}{section} % Number equations within sections (i.e. 1.1, 1.2, 2.1, 2.2 instead of 1, 2, 3, 4)
    % \numberwithin{figure}{section} % Number figures within sections (i.e. 1.1, 1.2, 2.1, 2.2 instead of 1, 2, 3, 4)
    % \numberwithin{table}{section} % Number tables within sections (i.e. 1.1, 1.2, 2.1, 2.2 instead of 1, 2, 3, 4)


%%%%%%%%%%%%%%%%%
% MATH COMMANDS %
%%%%%%%%%%%%%%%%%
% Macros mathematical delimiters
\usepackage{mathtools} % Extra math tools such as PairedDelimiter
% To get automatically sized delimiters follow the command with an asterisk
% e.g  \norm*{\sin\paren*{\frac{1}{2}}}
    \DeclarePairedDelimiter\paren{(}{)} % Parenthesis Macro
    \DeclarePairedDelimiter\bra{[}{]} % Bracket Macro
    \DeclarePairedDelimiter\curly{\{}{\}} %Curly Brackets
    \DeclarePairedDelimiter\abs{\lvert}{\rvert} % Single vertical lines, absolute values
    \DeclarePairedDelimiter\norm{\lVert}{\rVert} % Double vertical values, norm
    \DeclarePairedDelimiter\floor{\lfloor}{\rfloor} % Floor deilimiters
    \DeclarePairedDelimiter\ceil{\lceil}{\rceil} % Ceiling delimiters

%Other useful macro symbols
    \newcommand{\grad}{\nabla} % For gradients
    \newcommand{\deriv}[2]{\frac{d #1}{d #2}} %Derivative macro
    \newcommand{\pderiv}[2]{\frac{\partial #1}{\partial #2}} % Partial derivative macro
    \newcommand{\bigO}{\mathcal{O}} % For fancy Big O notation
    \newcommand{\gm}[1] {\guillemotleft #1\guillemotright} %$ pretty gimollet <<sth>>

%%%% Vector symbols
% Really useful when doing vector calculus or machine learning
    % \newcommand{\vx}{\bar{x}}
    % \newcommand{\vv}{\bar{v}}
    % \newcommand{\vw}{\bar{w}}
    % \newcommand{\vu}{\bar{u}}
    % \newcommand{\vy}{\bar{y}}
    % \newcommand{\vt}{\bar{\theta}}

%%%%%%%%%%%%%%%%%%%%%
% REVISION COMMANDS %
%%%%%%%%%%%%%%%%%%%%%
\usepackage[colorinlistoftodos]{todonotes} % useful for leaving todonotes

% Todo Commands \hightodo, \lowtodo
    \newcommand{\hightodo}[1]{\todo[inline,backgroundcolor=red!90!yellow!80!white]{\textbf{\userId}: #1}}
    \newcommand{\lowtodo}[1]{\todo[inline,backgroundcolor=blue!30!white]{\textbf{\userId}: #1}}

% Tip box environments
\usepackage{tcolorbox} % Color boxes for comment
\newtcolorbox{tip}{colback=blue!5!white,colframe=blue!75!black} % Vanilla tip box
\newtcolorbox{tipt}[1]{colback=blue!5!white,colframe=blue!75!black,fonttitle=\bfseries,title=#1} % Tip box with title

% Thick strike out line
    \newcommand{\soutthick}[1]{%
        \renewcommand{\ULthickness}{3pt}%
           \sout{#1}%
        \renewcommand{\ULthickness}{.4pt}% Resetting to ulem default
    }

%%%%%%%%%%%%
% QUOTE    %
%%%%%%%%%%%%

\makeatletter
\newenvironment{chapquote}[2][2em]
  {\setlength{\@tempdima}{#1}%
   \def\chapquote@author{#2}%
   \parshape 1 \@tempdima \dimexpr\textwidth-2\@tempdima\relax%
   \itshape}
  {\par\normalfont\hfill--\ \chapquote@author\hspace*{\@tempdima}\par\bigskip}
\makeatother



% Define blank page

\newcommand\blankpage{%
    \mbox{}%
    \thispagestyle{empty}%
    \newpage%
}

% Epigraph left side (good for quotes or dedications)

\usepackage{epigraph}

\setlength\epigraphwidth{8cm}
\setlength\epigraphrule{0pt}
\makeatletter
\patchcmd{\epigraph}{\@epitext{#1}}{\itshape\@epitext{#1}}{}{}
\makeatother


%%%%%%%%%%%%
% UNITS    %
%%%%%%%%%%%%

\DeclareSIUnit{\EUR}{\text{\euro}}
\sisetup{
  per-mode = symbol,
  % inter-unit-product = \ensuremath{{}\cdot{}}, % Cdot between units
}